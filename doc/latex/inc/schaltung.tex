
\chapter{Schaltungsentwurf}


\section{Beschreibung}


\section{Auslegung}



\subsubsection{Spannungsteiler mit RC-Glied}

\begin{figure}[htb]%
	\centering
	\begin{tikzpicture}
		\ctikzset {label/align = straight}
		\draw[color=black, thick]
			(0,0) to [short,o-] (1,0)
				to [R,-*,l^=$R_1$] (3,0)
				to [R,-*,l^=$R_3$] (5,0)
				to [short,-o] (7,0)
			(3,0) to [R,-*,l^=$R_2$] (3,-2)
			(5,0) to [C,-*,l^=$C$] (5,-2)
			(0,-2) to [short,o-o] (7,-2);
			
		\draw[->, thick] (0,-0.2) -- node [auto,swap] {$U_\mrm{in}$} (0,-1.8);
		\draw[->, thick] (7,-0.2) -- node [auto] {$U_\mrm{out}$} (7,-1.8);
	\end{tikzpicture}
	\caption{Spannungsteiler mit RC-Glied}%
	\label{fig:VDivRC}%
\end{figure}		

Die Schaltung aus \figref{fig:VDivRC} (ideale Spannungsquelle am Eingang, offene Klemmen am Ausgang) stellt ein PT$_1$-Glied mit der stationären Verstärkung
\begin{align*}
	\frac{R_2}{R_1 + R_2}
\end{align*}
und der Zeitkonstante
\begin{align*}
	T = C \cdot \left( \frac{R_1 R_2}{R_1 + R_2} + R_3 \right)
\end{align*}
dar.



\section{Verbesserungen für weitere Versionen}

\begin{itemize}
	\item Hall-Sensor muss nicht über OpAmp geleitet werden. Dafür wäre ein \valunit{5}{V}-toleranter Eingang empfehlenswert.
\end{itemize}



\section{Fehler und Hinweise zur Bestückung}

\begin{itemize}
	\item Bei dem Bestückungsdruck auf der Unterseite der aktuellen Platine sind R20 und R66 vertauscht.
	\item Die Pins 3 und 4 von P9 (Stift"|leiste Hall-Sensor) sind auf der Unterseite der Platine mit eine \valunit{10}{k\Omega}-Widerstand (R68) zu verbinden.
	\item Die Kondenstatoren C13 und C19 sind nicht zu bestücken. Diese können zu einem Einkoppeln von Störungen des Fahrtenreglers auf die ucboard-Masse führen. (Bei den ursprünglichen Fahrtenreglern wurden keine Probleme festgestellt, aber bei dem Fahrtenregler, der für eine höhere Spannung gekauft wurde, haben sich extreme Probleme gezeigt.)
\end{itemize}

