

\chapter{Flashen}


\textbf{Hinweis:} Bisher wird hier nur die Variante des Flashens unter Windows mit dem "`STM32 ST-Link Utility"' beschrieben.

\section{Vorbereitung}

"`STM32 ST-Link Utility"' von der STM-Homepage herunterladen und installieren (beinhaltet auch die Treiber). 

\textbf{Hinweis:} Die könnte schon problematisch sein, wenn man den ST-Link/V2 zum Debuggen verwendet will, da damit auch ein spezieller Treiber installiert wird, für gbd aber ein generischer Treiber benötigt wird. (Quelle: Forum, quergelesen. Müsste noch geklärt werden.)



\section{Flashen}

\begin{enumerate}
	\item ST-Link/V2 an Platine anschließen. Fahrzeug einschalten.
	\item "`STM32 ST-Link/V2 Utility"' öffnen.
	\item Auf "`Connect"'-Icon (drittes von links) klicken. Es sollten die Daten des Mikrocontrollers angezeigt werden.
	\item Auf "`Open File"'-Icon (erstes von links) klicken und bin-Datei auswählen.
		\begin{itemize}
			\item Im Repository unter \verb|fw_releases\| liegen die "`offiziellen"' Versionen.
		\end{itemize}
	\item Auf "`Program Verify"'-Icon (sechstes von links) klicken.
		\begin{itemize}
			\item Start-Adresse ist ist \verb|0x08000000|. (Sollte voreingestellt sein.)
		\end{itemize}
\end{enumerate}



